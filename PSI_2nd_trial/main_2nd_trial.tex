%--------------------
% Packages
% -------------------
\documentclass[11pt,a4paper]{article}
\usepackage[utf8x]{inputenc}
\usepackage[T1]{fontenc}
%\usepackage{gentium}
\usepackage{mathptmx} % Use Times Font


\usepackage{comment}
\usepackage[pdftex]{graphicx} % Required for including pictures
\usepackage[pdftex,linkcolor=black,pdfborder={0 0 0}]{hyperref} % Format links for pdf
\usepackage{calc} % To reset the counter in the document after title page
\usepackage{enumitem} % Includes lists

\frenchspacing % No double spacing between sentences
\linespread{1.2} % Set linespace
\usepackage[a4paper, lmargin=0.1666\paperwidth, rmargin=0.1666\paperwidth, tmargin=0.1111\paperheight, bmargin=0.1111\paperheight]{geometry} %margins
%\usepackage{parskip}

\usepackage[all]{nowidow} % Tries to remove widows
\usepackage[protrusion=true,expansion=true]{microtype} % Improves typography, load after fontpackage is selected

\usepackage{lipsum} % Used for inserting dummy 'Lorem ipsum' text into the template

\usepackage[export]{adjustbox}
\usepackage{plantuml}
\usepackage{float}
\usepackage{indentfirst}
\usepackage{subcaption}

\usepackage{graphicx}

%\usepackage{etoolbox} % Adds '\clearpage' to every '\section command' (newpage).
%\pretocmd{\section}{\clearpage}{}{} 

\newcommand{\inputdiagram}[1]{\input{Diagrams/out/#1}}
\newcommand{\textwidthdiagram}[2][1]{%
  \resizebox{#1\textwidth}{!}{\inputdiagram{#2}}%
}

%-----------------------
% Set pdf information and add title, fill in the fields
%-----------------------
\hypersetup{ 	
pdfsubject = {Program Systems Engineering},
pdftitle = {itssoover},
pdfauthor = {Ignas Časas, Mykolas Marius Budrys, Augustas Kniška}
}

%-----------------------
% Begin document
%-----------------------
\begin{document} %All text i dokumentet hamnar mellan dessa taggar, allt ovanför är formatering av dokumentet

\begin{titlepage}
    \centering
    % Remove page numbering from title
    \thispagestyle{empty}
    
    % University name
    {\Large VILNIAUS UNIVERSITETAS\\
    Matematikos ir informatikos fakultetas}\par
    
    \vspace{3cm} % vertical space
    
    % Title of the work
    {\Large 2\textsuperscript{nd} Laboratory Work}\par
    \vspace{0.5cm}
    {\Large \textbf{itssoover}}\par
    {\Large \textbf{Design and Implementation}}\par
    
    \vspace{3cm}
    
    % Authors
    {\large
    Ignas Časas\\
    Mykolas Marius Budrys\\
    Augustas Kniška
    }\par
    
    \vspace{8cm}
    
    % Bottom of the page
    {\large
    Matematikos ir informatikos fakultetas\\
    Vilniaus universitetas\\
    Lietuva
    }\par
    
    \vfill

    \large 2025
    
\end{titlepage}

\tableofcontents
\newpage

\section{Context}
Our web-based application offers a variety of cognitive games. They are designed to improve cognitive skills like memory, attention, reaction time and math problem-solving skills. The platform is designed to be  accessible from anywhere with only a browser, no additional software is needed. Platform's games have different difficulties, users are able to choose the game's complexity based on their preference. Registered users have the luxury of tracking their own progress throughout the lifespan of their account. The games are designed to be engaging; however, the priority is to measure and boost cognitive improvement rather than just entertainment.

\subsection{Technology Stack}
\begin{itemize}
    \item \textbf{Frontend:} Built with Blazor for interactive C\#-based UI.
    \item \textbf{Backend:} .NET (C\#) Web API for business logic, user authentication, and statistics tracking.
    \item \textbf{Database:} SQLite for lightweight storage of user data and game progress.
\end{itemize}

\subsection{Content \& Accessibility Boundaries}
\begin{itemize}
    \item Games focus on cognitive improvement, not entertainment or storytelling.
    \item No complex mechanics (e.g., multiplayer or 3D).
    \item Web-based only—no mobile apps, downloads, or third-party integrations.
\end{itemize}

\subsection{Functionality \& Monetization}
\begin{itemize}
    \item Single-player only; no social features.
    \item No in-game purchases or microtransactions.
\end{itemize}



\section{Logical View}


\subsection{Class Diagram}

\begin{figure}[H]
     \centering
     \begin{adjustbox}{width=0.92\paperwidth,center}        \inputdiagram{class_diagram.tex}
     \end{adjustbox}
     \caption{Class Diagram}
     \label{fig:class_diagram}
\end{figure}

Figure~\ref{fig:class_diagram} illustrates the real class structure of the repository. Each class is a representation of an actual class in the project. The code-base is separated into three parts: client, server, and shared components. Not all architecture details are depicted in the diagram, but major elements are shown that illustrate the main underlying implementation. 


Client-side page components inherit from BlazerComponentBase. AccountPage and StatisticsPage are for tracking cognitive progress, SettingsPage is for the user's privacy settings. MathGamePage, SudokuGamePage, PairMatchingGamePage, and AimTrainerPage are for game-specific functionality. TimerService is used by each game to track time. All the pages are located in the "Client/Pages" folder, while TimeService is in the "Client/Services".


Server-side has three main parts. First, are the controllers like AccountScoreController, MathGameController, etc. They help to handle the incoming requests from the user side. Controllers are saved in the "Server/Controllers" folder. Second are the services like MathGameService, SudokuService, AimTrainerService, PairUpService, AccountScoreService and UserService. They handle the backend logic. Services are located in the "Server/Services" folder. Lastly, the Repositories like ScoreRepository and UserRepository. They handle the interactions with the database. Repositories are located in the "Server/Repositories" folder.


The shared resources mainly help by providing data models so that the server side and client side could easily interact. From the client class diagram, these would be the UserScoreDto<T> and AverageScoreDtomodels. They are located in the "Shared/Models" folder.

\subsection{Client States}

The relation between the components shown in the state diagrams and classes is explained in the Component section of the document.

\begin{figure}[H]
    \centering
    \begin{minipage}[b]{0.59\textwidth}
        \centering
        \textwidthdiagram{score_fetching_state.tex}
        \caption{AimTrainerPage Score Fetching}
        \label{fig:score_fetching_state}
    \end{minipage}
    \hfil
    \begin{minipage}[b]{0.4\textwidth}
        \centering
        \textwidthdiagram{user_authentication_state.tex}
        \caption{AccountAuthStateProvider State}
        \label{fig:user_authentication_state}
    \end{minipage}
\end{figure}

Figure~\ref{fig:score_fetching_state} outlines the asynchronous score
retrieval process by the AimTrainerPage component. Other Page components in
the Game Pages subsystem manage state in the same way. The state enters in
a "Loading Scores" state where a background call (GetUserHighScoreAsync())
fetches data. Depending on the outcome, the state branches into one of three
paths: it moves to "No Scores Found" if no score is returned, "Encountered
Error" if an error occurs, or "Showing Scores" if the scores are successfully
loaded. The User is shown different UI elements based on the AimTrainerPage
state. This structure ensures that the application handles different scenarios
gracefully while providing clear feedback to the user.

The AccountAuthStateProvider State diagram presented in
Figure~\ref{fig:user_authentication_state} defines the different possible
states of user authentication kept by AccountAuthStateProvider. The
component begins in an "Unauthenticated" state. When an external call
to mark the user as authenticated is made the component changes it's
internal state to "Authenticated". If a MarkUserAsLoggedOut call is made
and the AccountAuthStateProvider is in the "Authenticated" state the
method then reverts the component back to the "Unauthenticated" state. The
AccountAuthStateProvider component is user by the AccoutPage component. This
model clearly delineates the transitions between being logged in and logged
out, ensuring that the system maintains a secure and predictable user session
management process.

\subsection{Game States}

These state diagrams are made to represent the internal state kept by
the components of the GamePage when the game is being played.

\begin{figure}[H]
    \centering
    \begin{minipage}[b]{0.48\textwidth}
        \centering
        \textwidthdiagram{math_state.tex}
        \caption{MathGamePage State Representation}
        \label{fig:math_state}
    \end{minipage}
    \hfil
    \begin{minipage}[b]{0.48\textwidth}
        \centering
        \textwidthdiagram{aim_trainer_state.tex}
        \caption{AimTrainerPage State Representation}
        \label{fig:aim_trainer_state}
    \end{minipage}
\end{figure}

In Figure~\ref{fig:aim_trainer_state} we model our \textit{Aim Training} game. The
game starts by displaying a target, and upon the user's click, it transitions
to a state where the input is processed. If there is remaining time (i.e.,
[time\_left>0]), a new target is generated, looping back to the display state;
otherwise, the game ends. The \textit{ProcessingInput} state handles updating both
the failure count (if the target was missed) and the score.

Figure~\ref{fig:math_state} represents our arithmetic based \textit{Math Game}. It
begins by displaying an equation to the user. When an answer is submitted, the
system transitions to a validation state. The validation state is responsible
for both making an asynchronous call to the back-end for answer validation
and increasing the kept score if the answer was valid. If time remains,
a new question is generated and the process repeats; if not, the game
terminates. Modeling our game in this way highlights the cyclic gameplay
loop that emphasizes both continuous engagement and time-bound gameplay,
ensuring the game session concludes when the allotted time is exhausted.

\begin{figure}[H]
    \centering
    \begin{minipage}[b]{0.48\textwidth}
        \centering
        \textwidthdiagram{sudoku_state.tex}
        \caption{SudokuPage State Representation}
        \label{fig:sudoku_state}
    \end{minipage}
    \hfil
    \begin{minipage}[b]{0.48\textwidth}
        \centering
        \textwidthdiagram{pair_up_state.tex}
        \caption{PairUpPage State Representation}
        \label{fig:pair_up_state}
    \end{minipage}
\end{figure}

Figure~\ref{fig:sudoku_state} (The \textit{Sudoku} state diagram) shows the different states and transitions of the SudokuPage component. The game begins in the Idle state and. User action can then trigger two different transitions: \textit{select cell} and \textit{submit}. The CellSelected state represents the state where one of the sudoku cells were selected by the user, the user can then trigger transitions by either selecting another cell, deselecting a cell completely, entering a value (number) inside the cell or submiting the current board for validation. The ValidatingBoard state represents the state in which the SudokuPage is communicating asynchronously with the back-end API in order to check if the submitted solution is valid. If the submission was invalid the component transitions back to the idle state, otherwise the game ends.

The \textit{Pair Up} diagram (Figure~\ref{fig:pair_up_state}) details the
states of our memory-based matching game. The game starts with all cards being
shown to the user. The user selects his first card (which the user cannot
unselect), then his second card, prompting the system to evaluate whether the
selected pair matches. If there are still unpaired cards remaining, the game cycles
back to display the cards; if all pairs are successfully matched,
the game ends. This design effectively captures the iterative nature of this
game while clearly defining the conditions for continuation and termination.


\subsection{Information Model}

\begin{figure}[H]
    \centering
    \includegraphics[width=\textwidth]{Diagrams/out/PNG/ER_Diagram_fix.drawio.png}
    \caption{ER Diagram}
    \label{fig:ER_diagram}
\end{figure}

In Figure~\ref{fig:ER_diagram}, the database structure of the system is displayed using an Entity-Relationship (ER) diagram. The diagram represents the entities (tables), their attributes, and the relationships between them. The database consists of five tables, each representing a key component of the system: User, AimTrainerScore, MathGameScore, PairUpScore, SudokuScore. Each score table has a direct relationship with the User table. Since a user can play multiple games and record multiple scores, the relationships are one-to-many. The Timestamp attribute is recorded directly within each score table.


\section{Development View}

\subsection{Package Diagram}

\begin{figure}[H]
    \centering
    \textwidthdiagram{Package_Diagram.latex}
    \caption{Package Diagram} 
    \label{fig:united_package}
\end{figure}

Figure~\ref{fig:united_package} provides an overview of how the codebase organization. For clarity and less abstraction, this diagram has some parts of the system that were not mentioned in the class diagram, such as "Shared/Enums", "Client/Components", and "Server/Database". This decision was made in order to simplify the class diagram while maintaining the key logical elements, but for the deployment diagram purposes they were added to provide more detail.

Similarly, as in the class diagram, the package diagram is split into Shared, Server, and Client folders. Not all folders are listed in the diagram, but the most important ones are. This was done to reduce clutter and make it easier to understand visually.

The Shared package highlights common elements across the project, such as data models and enumerations. A lot of parts in the client and server use these folders.

The server has controllers that await the client side requests, the controllers interact with services to perform certain tasks, such as saving a game record into the database. In this case, services will access the repositories' folder, which in turn will save a record in the database.

The Client has pages, components, and services. Pages interact with components and services, services send the requests to the server controllers.

The arrows indicate dependencies and illustrate how functionality is physically imported from one package to another.

\subsection{Component Diagrams}

\begin{figure}[H]
    \centering
    \textwidthdiagram{server_component_diagram.latex}
    \caption{Server Component Diagram}
    \label{fig:server_component_diagram}
\end{figure}

Figure~\ref{fig:server_component_diagram} illustrates various components which make up the server package. Controllers provide API endpoints to the corresponding services using interfaces as a mediator. These services then implement the needed server functions, such as game mechanics, user score tracking and account management, sometimes using other necessary services or repositories as needed. User and score data access is handled through repositories that abstract database operations.


The following package folder list helps us identify which class diagram components correspond to the components in figure~\ref{fig:server_component_diagram}:
\begin{enumerate}[label=\textbf{\arabic*.}, ref=\arabic*]
    \item \textbf{Services}
        \begin{enumerate}[label=\textbf{\alph*.}, ref=\theenumi.\alph*]
            \item \textbf{User Services Subsystem}
                \begin{itemize}
                    \item \textbf{Component:} \texttt{UserTrackingService}
                        \begin{itemize}
                            \item \textbf{Corresponding Class:} \texttt{UserTrackingService}
                        \end{itemize}
                    \item \textbf{Component:} \texttt{UserService}
                        \begin{itemize}
                            \item \textbf{Corresponding Class:} \texttt{UserServices}
                        \end{itemize}
                    \item \textbf{Component:} \texttt{AccountScoreService}
                        \begin{itemize}
                            \item \textbf{Corresponding Class:} \texttt{accountScoreService}
                        \end{itemize}
                \end{itemize}
            \item \textbf{Game Services Subsystem}
                \begin{itemize}
                    \item \textbf{Component:} \texttt{AimTrainerService}
                        \begin{itemize}
                            \item \textbf{Corresponding Class:} \texttt{AimTrainerService}
                        \end{itemize}
                    \item \textbf{Component:} \texttt{SudokuService}
                        \begin{itemize}
                            \item \textbf{Corresponding Class:} \texttt{SudokuService}
                        \end{itemize}
                    \item \textbf{Component:} \texttt{PairUpService}
                        \begin{itemize}
                            \item \textbf{Corresponding Class:} \texttt{PairUpService}
                        \end{itemize}
                \end{itemize}
            \item \textbf{Math Game Services Subsystem}
                \begin{itemize}
                    \item \textbf{Component:} \texttt{MathGameServices}
                        \begin{itemize}
                            \item \textbf{Corresponding Class:} \texttt{MathGameService}
                        \end{itemize}
                    \item \textbf{Component:} \texttt{MathGenerationService}
                        \begin{itemize}
                            \item \textbf{Corresponding Class:} (Not in Class Diagram)
                        \end{itemize}
                    \item \textbf{Component:} \texttt{MathCalculationService}
                        \begin{itemize}
                            \item \textbf{Corresponding Class:} (Not in Class Diagram)
                        \end{itemize}
                    \item \textbf{Component:} \texttt{MathGameScoreboardService}
                        \begin{itemize}
                            \item \textbf{Corresponding Class:} (Not in Class Diagram)
                        \end{itemize}
                \end{itemize}
        \end{enumerate}
    \item \textbf{Controllers}
        \begin{enumerate}[label=\textbf{\alph*.}, ref=\theenumi.\alph*]
            \item \textbf{User Controllers Subsystem}
                \begin{itemize}
                    \item \textbf{Component:} \texttt{UserController}
                        \begin{itemize}
                            \item \textbf{Corresponding Class:} \texttt{UserController}
                        \end{itemize}
                    \item \textbf{Component:} \texttt{AccountScoreController}
                        \begin{itemize}
                            \item \textbf{Corresponding Class:} \texttt{AccountScoreController}
                        \end{itemize}
                \end{itemize}
            \item \textbf{Game Controllers Subsystem}
                \begin{itemize}
                    \item \textbf{Component:} \texttt{AimTrainerController}
                        \begin{itemize}
                            \item \textbf{Corresponding Class:} \texttt{AimTrainerController}
                        \end{itemize}
                    \item \textbf{Component:} \texttt{SudokuController}
                        \begin{itemize}
                            \item \textbf{Corresponding Class:} \texttt{SudokuController}
                        \end{itemize}
                    \item \textbf{Component:} \texttt{PairUpController}
                        \begin{itemize}
                            \item \textbf{Corresponding Class:} \texttt{PairUpController}
                        \end{itemize}
                    \item \textbf{Component:} \texttt{MathGameController}
                        \begin{itemize}
                            \item \textbf{Corresponding Class:} \texttt{MathGameController}
                        \end{itemize}
                \end{itemize}
        \end{enumerate}
    \item \textbf{Repositories Folder}
        \begin{itemize}
            \item \textbf{Component:} \texttt{User Repository}
                \begin{itemize}
                    \item \textbf{Corresponding Class:} \texttt{UserRepository}
                \end{itemize}
            \item \textbf{Component:} \texttt{Score Repository}
                \begin{itemize}
                    \item \textbf{Corresponding Class:} \texttt{ScoreRepository}
                \end{itemize}
        \end{itemize}
    \item \textbf{Database Folder}
        \begin{itemize}
            \item \textbf{Component:} \texttt{UserDbContext}
                \begin{itemize}
                    \item \textbf{Corresponding Class:} (Not in Class Diagram)
                \end{itemize}
        \end{itemize}
\end{enumerate}

\begin{figure}[H]
    \centering
     \begin{adjustbox}{width=0.92\paperwidth,center}
         \inputdiagram{client_components.tex}
     \end{adjustbox}
    \caption{Client Package Component Diagram}
    \label{fig:client_components}
\end{figure}

Figure~\ref{fig:client_components} visualizes the components inside the Client package.  The pages
are divided into two subsytems: User Pages - for handling client-side
authentication and account managment related views, Game Pages - for pages
containing game implementations. The Page components interface with components
from both the Account Services and Game Services subsystems.  The service
components are responsible for storing game or account related data locally,
substituting parts of game logic implementation or making HTTP requests to
the Web API. Hence they require HTTP API Endpoints for their functionality.

The following outlines the mapping between components from a client component diagram and corresponding classes in a class diagram:
\begin{enumerate}[label=\textbf{\arabic*.}, ref=\arabic*]
    \item \textbf{Services}
        \begin{enumerate}[label=\textbf{\alph*.}, ref=\theenumi.\alph*]
            \item \textbf{Timer Service}
                \begin{itemize}
                    \item \textbf{Component:} \texttt{TimerService}
                        \begin{itemize}
                            \item \textbf{Equivalent Class:} \texttt{TimerService}
                        \end{itemize}
                \end{itemize}
            \item \textbf{Account Services Subsystem}
                \begin{itemize}
                    \item \textbf{Component:} \texttt{AccountScoreService}
                        \begin{itemize}
                            \item \textbf{Equivalent Class:} \texttt{AccountScoreService}
                        \end{itemize}
                    \item \textbf{Component:} \texttt{AccountService}
                        \begin{itemize}
                            \item \textbf{Equivalent Class:} (Not in Class Diagram)
                        \end{itemize}
                    \item \textbf{Component:} \texttt{AccountAuthStateProvider}
                        \begin{itemize}
                            \item \textbf{Equivalent Class:} (Not in Class Diagram)
                        \end{itemize}
                \end{itemize}
            \item \textbf{Game Services Subsystem}
                \begin{itemize}
                    \item \textbf{Component:} \texttt{SudokuService}
                        \begin{itemize}
                            \item \textbf{Equivalent Class:} \texttt{SudokuService}
                        \end{itemize}
                    \item \textbf{Component:} \texttt{MathGameService}
                        \begin{itemize}
                            \item \textbf{Equivalent Class:} \texttt{MathGameService}
                        \end{itemize}
                    \item \textbf{Component:} \texttt{PairUpService}
                        \begin{itemize}
                            \item \textbf{Equivalent Class:} \texttt{PairUpService}
                        \end{itemize}
                    \item \textbf{Component:} \texttt{AimTrainerService}
                        \begin{itemize}
                            \item \textbf{Equivalent Class:} \texttt{AimTrainerService}
                        \end{itemize}
                \end{itemize}
        \end{enumerate}
    \item \textbf{Pages}
        \begin{enumerate}[label=\textbf{\alph*.}, ref=\theenumi.\alph*]
            \item \textbf{User Pages Subsystem}
                \begin{itemize}
                    \item \textbf{Component:} \texttt{ScorePage}
                        \begin{itemize}
                            \item \textbf{Equivalent Class:} \texttt{StatisticsPage}
                        \end{itemize}
                    \item \textbf{Component:} \texttt{AccountPage}
                        \begin{itemize}
                            \item \textbf{Equivalent Class:} \texttt{AccountPage}
                        \end{itemize}
                    \item \textbf{Component:} \texttt{SettingsPage}
                        \begin{itemize}
                            \item \textbf{Equivalent Class:} \texttt{SettingsPage}
                        \end{itemize}
                \end{itemize}
            \item \textbf{Game Pages Subsystem}
                \begin{itemize}
                    \item \textbf{Component:} \texttt{AimTrainerPage}
                        \begin{itemize}
                            \item \textbf{Equivalent Class:} \texttt{AimTrainerPage}
                        \end{itemize}
                    \item \textbf{Component:} \texttt{SudokuPage}
                        \begin{itemize}
                            \item \textbf{Equivalent Class:} \texttt{SudokuPage}
                        \end{itemize}
                    \item \textbf{Component:} \texttt{MathGamePage}
                        \begin{itemize}
                            \item \textbf{Equivalent Class:} \texttt{MathGamePage}
                        \end{itemize}
                    \item \textbf{Component:} \texttt{PairUpPage}
                        \begin{itemize}
                            \item \textbf{Equivalent Class:} \texttt{PairMatchingGamePage}
                        \end{itemize}
                \end{itemize}
        \end{enumerate}
\end{enumerate}

\section{Process View}

\subsection{Sequence Diagrams}

\begin{figure}[H]
    \centering
     \begin{adjustbox}{width=0.92\paperwidth,center}
         \inputdiagram{aim_trainer_page_sequence.tex}
     \end{adjustbox}
     \caption{\textit{Aim Trainer} Page Loading Sequence}
    \label{fig:aim_trainer_page_sequence}
\end{figure}
In order to visualize the way in which score fetching works for all of
our games we will do a case analysis on the \textit{AimTrainerPage} component.
Figure~\ref{fig:aim_trainer_page_sequence} illustrates the process flow
when a player opens the Aim Trainer page. Initially, the page immediately
requests the username from the authentication provider component. After
this, an initial page is shown to the user with an icon to indicate the fact
that scores are being loaded. Once the username is retrieved and confirmed
not to be null, the page initiates an asynchronous call to the client-side
service to fetch the player's high score. The client service makes an HTTP GET
request to an endpoint provided by the Aim Trainer controller, which, in turn,
invokes the server-side service. The server service interacts with the Score
Repository and User Database Context to locate the corresponding user and
their scores. Depending on whether the user and score exist, the repository
returns either a list of scores or an empty list. The server service then
filters the result to extract the highest score, and this filtered result
is passed back up the chain and sent back as JSON (if a score was found)
and an empty response otherwise. After parsing and unwrapping the data on
the client side, the Aim Trainer page updates its display, showing either the
retrieved high score or a "Not Found" message if no score was found. Finally,
the updated view is presented to the player. This is the way all of our
games implement the score fetching functionality.

\begin{figure}[H]
    \centering
     \begin{adjustbox}{width=0.92\paperwidth,center}
        \inputdiagram{aim_trainer_set_score_sequence.tex}
     \end{adjustbox}
    \caption{Aim Trainer Score Setting Sequence}
    \label{fig:aim_trainer_set_score_sequence}
\end{figure}

This sequence diagram (Figure~\ref{fig:aim_trainer_set_score_sequence})
is an example of the process initiated by finishing a play session in one
our games \textbf{WHILE AUTHENTICATED} (otherwise no such process occurs
on the client side). Once again we analyze only the AimTrainerPage, however
other pages operate in a similar way. Once the game ends, the page calls the
client-side service to save the player's score asynchronously. The client
service wraps the score within a DTO and forwards it via an HTTP POST request
to an API endpoint provided by the Aim Trainer controller. The controller
then uses the server-side service component to process the request. The
server-side service unwraps the DTO and uses an interface provided by the
Score Repository, which attempts to find the corresponding user within the
given database context.  If the user is found, the repository attempts to
add the score to the database; if an exception occurs during this process it
is caught by the controller and the controller decides the appropriate HTTP
error code to return (HTTP 500 for DB exceptions, or HTTP 400 when a user
is not found, indicating that a deceptive request might have been sent). If
the user is found, and the insertion process is successful the controller
returns a HTTP 200 OK code. Finally, the result is propagated back through
the client service to the Aim Trainer page, updating the UI accordingly.

\subsection{Activity Diagram}
\begin{figure}[H]
    \centering
     \begin{adjustbox}{width=0.92\paperwidth,center}
        \inputdiagram{leaderboard_activity.tex}
     \end{adjustbox}
    \caption{Leaderboard Loading Diagram}
    \label{fig:leaderboard_activity}
\end{figure}

To show the way in which leaderboard scores get loaded by the components in
the Game Pages subsystem we show an example of how the PairUpPage component
executes this functionality (Fig.~\ref{fig:leaderboard_activity}.  This process
is initiated by a Player navigating to the Pair Up game page. The PairUpPage
component then handles showing the Player the needed loading icons and making
an asynchronous call chain to fetch the leaderboards from the API in parallel.
The HTTP request is sent by the PairUpService(C) (C - indicating the client
side service) component.  The PairUpController component handles receiving
the request and passing it along to the PairUpService(S) (S - indicating
the server side service).  PairUpService(S) then receives the leaderboard
entries by using the ScoreRepository component and filters out the entries
of players which have chosen to hide their high scores.  It then returns the
entries to the controller which sends a response to the initial HTTP request.
The PairUpService(C) parses the response and throws an exception if it did
not receive a valid response.  The PairUpPage component then handles logic
for updating the final page view.

\section{Physical View}
\subsection{Deployment}
\begin{figure}[H]
    \centering
    \begin{adjustbox}{width=0.92\paperwidth,center}
        \inputdiagram{deployment_diagram.tex}
     \end{adjustbox}
    \caption{Deployment Diagram}
    \label{fig:deployment_diagram}
\end{figure}

Figure~\ref{fig:deployment_diagram} illustrates the deployment of the web
application, which consists of a client-side Blazor WebAssembly (WASM)
front-end app accessed from a browser and a back-end Web API hosted on
an Azure server. The client browser runs the provided .NET WASM runtime,
running Client.dll as an executable. The backend operates in a Linux (Ubuntu)
environment within Azure, where the .NET runtime hosts the Server.dll
executable. The Server.dll contains an entry point for the Web API, the
Client.dll contains an entry point for the Blazor App. Data is persisted using
an SQLite database (SQLite is built into the Server.dll). Communication
between the client and server occurs over HTTPS, ensuring secure data
transmission. This design enables a lightweight client while leveraging
cloud infrastructure for back-end processing and storage of user and game data.

\subsection{CI/CD}

\begin{figure}[H]
        \centering
        \textwidthdiagram[0.8]{CI.tex}
        \caption{CI Pipeline}
        \label{fig:CI_pipeline}
\end{figure}
Figure~\ref{fig:CI_pipeline} shows our current continuous integration pipeline that was set up using GitHub Actions. The pipeline is triggered when a pull request is made for the main branch. Firstly a runner is found that checks out the code and sets up the required environment. The project dependencies are then restored and the app is built. The tests are then run automatically. Based on the results of the tests, the pull request either gets marked as failing and is blocked from being merged into the main branch, or the PR gets marked as passing, giving a green light to merge after code review.

\begin{figure}[H]
    \centering
    \textwidthdiagram[0.8]{CD.tex}
    \caption{CD Pipeline}
    \label{fig:CD_pipeline}
\end{figure}
Our continuous deployment pipeline (Fig.~\ref{fig:CD_pipeline}) is also set up using GitHub Actions. It is triggered when a push/change happens on the main branch of the project. The pipeline is made up of two parts: the build and test job and the deployment job. At first, the code is checked out, the environment is set up, the project dependencies are restored, and the app is built. If the tests fail, deployment is aborted entirely; if they pass, the deployment job is started. This job handles downloading the built application from the previous job and uses Azure's deployment action to upload the artifacts to the Azure servers.

\section{Use Case View}

\begin{figure}[H]
    \centering
    \begin{minipage}[b]{0.48\textwidth}
        \centering
        \textwidthdiagram{use_case_account.tex}
        \caption{Account Use Case}
        \label{fig:use_case_account}
    \end{minipage}
    \hfil
    \begin{minipage}[b]{0.48\textwidth}
        \centering
        \textwidthdiagram{use_case_game_page.tex}
        \caption{Game Page}
        \label{fig:use_case_game_page}
    \end{minipage}
\end{figure}

Figure~\ref{fig:use_case_account} shows how both unauthenticated and authenticated players can use the account module. An already authenticated player can use it to log out, a unauthenticated player can use it to log in (authenticate) or to sign up if he wants to create a new account.

Figure~\ref{fig:use_case_game_page} contains the use cases for our Game Page components. A player can interact with the component to either start the game or to change the difficulty of the game before playing.

\begin{figure}[H]
    \centering
    \begin{minipage}[b]{0.48\textwidth}
        \textwidthdiagram{use_case_high_score.tex}
        \caption{High Score Module}
        \label{fig:use_case_high_score}
    \end{minipage}
    \hfil
    \begin{minipage}[b]{0.48\textwidth}
        \centering
        \textwidthdiagram{use_case_navigation.tex}
        \caption{Page Navigation}
        \label{fig:use_case_navigation}
    \end{minipage}
\end{figure}

Figure~\ref{fig:use_case_high_score} shows how authenticated players can use the high score module. Authenticated players can view the shortened version of their game statistics, view a visual graph of their game statistics, see a timeline of the games they've played or view their statistics filtered by game difficulty.

Figure~\ref{fig:use_case_navigation} outlines how players use the navigation components to navigate throughout the app. They can choose to go to the home page, which doubles as the game selection page, go to the account page, go to the statistics page, or go to the settings page.

\begin{figure}[H]
    \begin{minipage}[b]{0.48\textwidth}
        \centering
        \textwidthdiagram{use_case_game_selection.tex}
        \caption{Game selection}
        \label{fig:use_case_game_selection}
    \end{minipage}
    \hfil
    \begin{minipage}[b]{0.48\textwidth}
        \centering
        \textwidthdiagram{use_case_settings.tex}
        \caption{Account Settings}
        \label{fig:use_case_settings}
    \end{minipage}
\end{figure}
Figure~\ref{fig:use_case_game_selection} shows the use case of selecting a game from our game selection page. The user can choose to play the following games: Aim Trainer, Math Game, Pair Up, and Sudoku.

Figure~\ref{fig:use_case_settings} shows how authenticated players can interact with the settings components. They can modify their high score visibility by either setting it to public or setting it to private. Public scores can be viewed by all players on the leaderboard, while private ones are hidden.


\section{Traceability}

\begin{table}
\begin{tabular}{|p{0.35\textwidth}|p{0.6\textwidth}|}
\hline
\multicolumn{1}{|c|}{\textbf{Use Cases}}    & \multicolumn{1}{|c|}{\textbf{Coverage}}                   \\ \hline
\multicolumn{2}{|c|}{\textbf {Figure~\ref{fig:use_case_account} Account Use Case} }                                          \\ \hline
Log in& Handled by the UserService component (Server package)\newline
Handled by the AccountService component (Client package)\newline
Figure~\ref{fig:user_authentication_state} - \textit{Shows state transitions between authenticated and unauthenticated states, including login process}\\
\hline
Sign up& Handled by the UserService component (Server package)\newline
Handled by the AccountService component (Client package)\\
\hline
Log out& Handled by the UserService component (Server package)\newline
Handled by the AccountService component (Client package)\newline
Figure~\ref{fig:user_authentication_state} - \textit{Shows transition from authenticated to unauthenticated state during logout}\\
\hline
\multicolumn{2}{|c|}{\textbf {Figure~\ref{fig:use_case_game_page} Game Page Use Case} }                                      \\
\hline
Start game& Handled by the components from the Game Pages subsystem (Client package)\\
\hline
Change Difficulty& Handled by the components from the Game Pages subsystem (Client package)\\
\hline
\multicolumn{2}{|c|}{\textbf {Figure~\ref{fig:use_case_high_score} High Score Module Use Case} }                                         \\
\hline
View shortened game statistics& Figure~\ref{fig:score_fetching_state} - \textit{Handles asynchronous score retrieval and display process}\newline 
Figure~\ref{fig:client_components} - \textit{Details frontend components for displaying statistics}
\\
\hline
View graph of game statistics& Handled by AccountScoreService (Client package)\newline
Handled by AccountScoreService (Service package)\newline
StatisticsPage component(Client package)\\
\hline
View timeline of played games& Handled by AccountScoreService (Client package)\newline
Handled by AccountScoreService (Service package)\newline
StatisticsPage component(Client package)\\
\hline
View game statistics by difficulty& Handled by AccountScoreService (Client package)\newline
Handled by AccountScoreService (Service package)\newline
StatisticsPage component(Client package)\\
\hline
\end{tabular}
\end{table}

\begin{table}
\begin{tabular}{|p{0.35\textwidth}|p{0.6\textwidth}|}
\hline
\multicolumn{2}{|c|}{\textbf {Figure~\ref{fig:use_case_navigation} Page Navigation Use Case} }                          \\
\hline
Go to home page (game selection)& Provided by all \textit{Pages} components\newline
Figure~\ref{fig:client_components} - \textit{Shows UI routing components}\\
\hline
Go to account page& Provided by all \textit{Pages} components\newline
Handled by AccountPage component\newline
Figure~\ref{fig:client_components} - \textit{Shows UI routing components}\\
\hline
Go to statistics page& Provided by all \textit{Pages} components\newline
Handled by StatisticsPage component\newline
Figure~\ref{fig:client_components} - \textit{Shows UI routing components}\\
\hline
Go to settings page& Provided by all \textit{Pages} components\newline
Handled by SettingsPage component\newline
Figure~\ref{fig:client_components} - \textit{Shows UI routing components}\\
\hline
\multicolumn{2}{|c|}{\textbf {Figure~\ref{fig:use_case_game_selection} Game Selection Use Case} }           \\
\hline
Play "Aim Trainer"& AimTrainerPage component\newline
Figure~\ref{fig:aim_trainer_state} - \textit{Details game states and transitions for the Aim Trainer game}\\
\hline
Play "Math Game"& MathGamePage component\newline
Figure~\ref{fig:math_state} - \textit{Shows states and transitions for the Math Game}\\
\hline
Play "Pair Up"& PairUpPage component\newline
Figure~\ref{fig:pair_up_state} - \textit{Illustrates memory matching game states and logic flow}\\
\hline
Play "Sudoku"& SudokuPage component\newline
Figure~\ref{fig:sudoku_state} - \textit{Details puzzle game states and user interactions}\\
\hline
\multicolumn{2}{|c|}{\textbf {Figure~\ref{fig:use_case_settings} Account Settings Use Case} }   \\
\hline
Privatize High Score& AccountService component (Client package)\newline
SettingsPage component (Client package)\newline
UserService component (Server package)\\
\hline
Make High Score Public& AccountService component (Client package)\newline
SettingsPage component (Client package)\newline
UserService component (Server package)\\
\hline
\end{tabular}

\vspace{8pt} % value to increase the gap

\textbf{Note:} The traceability table assumes some coverage is implicit and can be further deduced by following interface relationships that are shown in the Client component diagram (Fig.~\ref{fig:client_components}) and the Server component diagram (Fig.\ref{fig:server_component_diagram}). 
\end{table}





%\lipsum[1-3]



\end{document}
