%--------------------
% Packages
% -------------------
\documentclass[11pt,a4paper]{article}
\usepackage[utf8x]{inputenc}
\usepackage[T1]{fontenc}
%\usepackage{gentium}
\usepackage{mathptmx} % Use Times Font


\usepackage[pdftex]{graphicx} % Required for including pictures
\usepackage[pdftex,linkcolor=black,pdfborder={0 0 0}]{hyperref} % Format links for pdf
\usepackage{calc} % To reset the counter in the document after title page
\usepackage{enumitem} % Includes lists

\frenchspacing % No double spacing between sentences
\linespread{1.2} % Set linespace
\usepackage[a4paper, lmargin=0.1666\paperwidth, rmargin=0.1666\paperwidth, tmargin=0.1111\paperheight, bmargin=0.1111\paperheight]{geometry} %margins
%\usepackage{parskip}

\usepackage[all]{nowidow} % Tries to remove widows
\usepackage[protrusion=true,expansion=true]{microtype} % Improves typography, load after fontpackage is selected

\usepackage{lipsum} % Used for inserting dummy 'Lorem ipsum' text into the template

\usepackage{plantuml}
\usepackage{float}
\usepackage{indentfirst}
\usepackage{subcaption}

\usepackage{etoolbox} % Adds '\clearpage' to every '\section command' (newpage).
\pretocmd{\section}{\clearpage}{}{} 

\newcommand{\inputdiagram}[1]{\input{Diagrams/out/#1}}
\newcommand{\textwidthdiagram}[2][1]{%
  \resizebox{#1\textwidth}{!}{\inputdiagram{#2}}%
}

%-----------------------
% Set pdf information and add title, fill in the fields
%-----------------------
\hypersetup{ 	
pdfsubject = {Program Systems Engineering},
pdftitle = {itssoover},
pdfauthor = {Ignas Časas, Mykolas Marius Budrys, Augustas Kniška}
}

%-----------------------
% Begin document
%-----------------------
\begin{document} %All text i dokumentet hamnar mellan dessa taggar, allt ovanför är formatering av dokumentet

\begin{titlepage}
    \centering
    % Remove page numbering from title
    \thispagestyle{empty}
    
    % University name
    {\Large VILNIAUS UNIVERSITETAS\\
    Matematikos ir informatikos fakultetas}\par
    
    \vspace{3cm} % vertical space
    
    % Title of the work
    {\Large 2\textsuperscript{nd} Laboratory Work}\par
    \vspace{0.5cm}
    {\Large \textbf{itssoover}}\par
    {\Large \textbf{Design and Implementation}}\par
    
    \vspace{3cm}
    
    % Authors
    {\large
    Ignas Časas\\
    Mykolas Marius Budrys\\
    Augustas Kniška
    }\par
    
    \vspace{8cm}
    
    % Bottom of the page
    {\large
    Matematikos ir informatikos fakultetas\\
    Vilniaus universitetas\\
    Lietuva
    }\par
    
    \vfill

    \large 2025
    
\end{titlepage}

\section{Context}
 Some text and ideas are mentioned...
 ??????

\section{Logical View}

\subsection*{Class Diagram}

\begin{figure}[H]
    \centering
    \resizebox{0.8\textwidth}{!}
        \input{class_diagram.latex}
    \caption{Class diagram}
    \label{fig:class_diagram}
\end{figure}


All frontend pages (AccountPage, StatisticsPage, SettingsPage, MathGamePage, SudokuGamePage, PairMatchingGamePage, AimTrainerPage) inherit from BlazerComponentBase. AccountPage and StatisticsPage provide interfaces for cognitive progress tracking. SettingsPage allows users to modify their privacy settings. The game-specific pages (MathGamePage, SudokuGamePage, PairMatchingGamePage, AimTrainerPage) are each designed for a unique game experience, and every gaming page utilizes a TimerService to consistently manage countdowns.

Services such as MathGameService, SudokuService, AimTrainerService, PairUpService, AccountScoreService, and UserService encapsulate the application's business logic and handle data operations. Repositories like ScoreRepository and UserRepository are responsible for interacting with the database.

Controllers including AccountScoreController, AimTrainerController, MathGameController, PairUpController, SudokuController, and UserController act as intermediaries, exposing API endpoints to connect the front end with the back-end services. Data is consistently exchanged between the front end and back end using Data Transfer Objects (DTOs) such as UserScoreDto<T> and AverageScoreDto. 

\subsection*{State Diagram}
TODO: Mykolas

\section{Development View}

\subsection*{Package Diagram}
TODO: Augustas
...Package diagram

\subsection*{Component Diagram}
TODO: Augustas
...Component diagram

\section{Process View}

\subsection*{Sequence Diagram}
TODO: Mykolas
\subsection*{Activity Diagram}
TODO: Mykolas

\section{Physical View}
\subsection{Deployment}
\begin{figure}[H]
    \centering
    \textwidthdiagram[0.8]{deployment_diagram.tex}
    \caption{Deployment diagram}
    \label{fig:deployment_diagram}
\end{figure}

Figure~\ref{fig:deployment_diagram} illustrates the architecture of the web application,
which consists of a client-side Blazor WebAssembly (WASM) front-end and a back-end hosted on an Azure server. The client browser executes the .NET WASM
runtime, running Client.dll as an executable. The backend operates in a Linux
environment within Azure, where the .NET runtime hosts Server.dll. Data is
persisted using an SQLite database. Communication between the client and
server occurs over HTTPS, ensuring secure data transmission. This design
enables a lightweight client while leveraging cloud infrastructure for
back-end processing and storage of user and game data.

\subsection{CI/CD}

\begin{figure}[H]
        \centering
        \textwidthdiagram[0.8]{CI.tex}
        \caption{CI Pipeline}
        \label{fig:CI_pipeline}
\end{figure}
Figure~\ref{fig:CI_pipeline} focuses on our pull request (PR) validation
workflow. Triggered by any pull request targeting the main branch, this
pipeline starts by setting up a linux environment (latest Ubuntu) and
executing a series of steps: checking out the code, setting up the .NET
environment, restoring dependencies, building the solution, and running
tests. The critical decision point here assesses whether all tests pass. If
they do, the PR status is marked as passing, signaling readiness for review
and potential merge. If any tests fail, the pipeline marks the PR as failing
and prevents it from being merged, thus upholding code quality and integrity
before changes are integrated into the main branch.

\begin{figure}[H]
    \centering
    \textwidthdiagram[0.8]{CD.tex}
    \caption{CD Pipeline}
    \label{fig:CD_pipeline}
\end{figure}
Figure~\ref{fig:CD_pipeline} illustrates our CD pipeline using GitHub
Actions. The whole process is excecuted on a linux(latest Ubuntu)
environment. The process begins when a push to the main branch or a manual
trigger starts the pipeline. Within the "Build and Test Job" partition,
the pipeline checks out the repository, sets up the .NET environment,
restores dependencies, and builds the application. After running tests,
the workflow evaluates the outcome: if the tests succeed, it proceeds to
publish the application and upload the resulting artifact; if any test fails,
the deployment is aborted immediately. Following a successful build and test
phase, the "Deploy Job" partition retrieves the artifact from the previous
step and deploys it to Azure, ensuring that only verified code is promoted
to production.

\section{Use Case View}

\begin{figure}[H]
    \begin{minipage}[b]{0.48\textwidth}
        \centering
        \textwidthdiagram{use_case_account.tex}
        \caption{Account Use Case}
        \label{fig:use_case_account}
    \end{minipage}
    \hfil
    \begin{minipage}[b]{0.48\textwidth}
        \centering
        \textwidthdiagram{use_case_game_page.tex}
        \caption{Game Page}
        \label{fig:use_case_game_page}
    \end{minipage}
\end{figure}

Figure~\ref{fig:use_case_account} shows how players interact with the Account module.
Unauthenticated players can log in or sign up, while authenticated players can log out.

Figure~\ref{fig:use_case_game_page} outlines two key actions: starting a game and
changing the difficulty level. This design ensures that players have direct
control over initiating gameplay and tailoring the challenge to their
preferences.

\begin{figure}[H]
    \centering
    \begin{minipage}[b]{0.48\textwidth}
        \textwidthdiagram{use_case_high_score.tex}
        \caption{High Score Module}
        \label{fig:use_case_high_score}
    \end{minipage}
    \hfil
    \begin{minipage}[b]{0.48\textwidth}
        \centering
        \textwidthdiagram{use_case_navigation.tex}
        \caption{Page Navigation}
        \label{fig:use_case_navigation}
    \end{minipage}
\end{figure}

Figure~\ref{fig:use_case_high_score} details various ways for authenticated players to review game performance. Users
can view short game statistics, detailed graphs, timelines of past games, and
statistics filtered by difficulty. The module provides multiple perspectives to
help players track and analyze their performance.

Figure~\ref{fig:use_case_navigation} maps out the main paths for navigating through the website.
Players can easily switch between the home page (which doubles as the game
selection page), the account page, and the statistics page, ensuring a
user-friendly and integrated browsing experience.

\begin{figure}[H]
    \begin{minipage}[b]{0.48\textwidth}
        \centering
        \textwidthdiagram{use_case_game_selection.tex}
        \caption{Game selection}
        \label{fig:use_case_game_selection}
    \end{minipage}
    \hfil
    \begin{minipage}[b]{0.48\textwidth}
        \centering
        \textwidthdiagram{use_case_settings.tex}
        \caption{Account Settings}
        \label{fig:use_case_settings}
    \end{minipage}
\end{figure}
Figure~\ref{fig:use_case_game_selection} illustrates the available games that
players can choose from. Players can select one of four games: Aim Trainer,
Math Game, Pair Up, or Sudoku. This selection allows users to navigate to
a game page based on their preferences.

Figure~\ref{fig:use_case_settings} shows the settings available to
authenticated players. Users can choose to make their high scores public
or keep them private. This feature provides control over personal game
performance visibility, enhancing user privacy options.

\section{Traceability}

%\lipsum[1-3]



\end{document}
